%% %%%%%%%%%%%%%%%%%%%%%%%%%%%%%%%%%%%%%%%%%%%%%%%%%%%%%%%%%%%%%%%%%%%%%%%%%%%
%%
%%    author(s): RoboCupAtHome Technical Committee(s)
%%  description: Introduction - Competition
%%
%% %%%%%%%%%%%%%%%%%%%%%%%%%%%%%%%%%%%%%%%%%%%%%%%%%%%%%%%%%%%%%%%%%%%%%%%%%%%
\section{Competition}
The competition consists of 2 \emph{Stages} and the \iterm{Finals}.

In \iterm{Stage~I}, teams carry out a series of free-form demonstrations of their robot. These demonstrations are expected to be held in a daily life environment, and are evaluated based on the achievement of key concepts that have been agreed upon the Technical and Executive Committee of the year in question (see \refchap{chap:stage_I} for details). The more key concepts are acheived, the better the overall score of that team for \iterm{Stage~I}.

The best teams from \iterm{Stage~I} then advance to \iterm{Stage~II} which consists of story-driven, integrated tests which are much more difficult to carry out.

The competition ends with the \emph{Finals} where only the two highest ranked teams of each league compete to select the winner.
